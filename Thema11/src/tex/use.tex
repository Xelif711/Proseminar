\chapter{Verwandte Konzepte in anderen Sprachen}
\section{In C++}

Das Konzept der inneren Klassen wurde erstmals 1989 in C++ eingeführt.
Diese sind sehr ähnlich zu Javas heutigen statischen inneren Klassen, wobei der größte Unterschied in der Möglichkeit Javas besteht, auf private, für andere Klassen nicht einsehbare Attribute zuzugreifen, was in C++ nicht möglich ist.
Also sind Javas innere Klassen eine Art Erweiterung der inneren Klassen von C++.
In C++11 wurden die inneren Klassen dann um die Möglichkeit einer "'normalen"' inneren Klasse Javas,
 nämlich auch auf nicht-statische Attribute der äußeren Klasse zugreifen zu können, erweitert (\cite{Ellis2007}).

\section{In C\# und Visual Basic}

Derartige Strukturen existieren auch in C\# und Visual Basic, welche beide von Microsoft produzierte Weiterentwicklungen von C++ sind.
Damit folgen diese auch fast den gleichen Regeln wie die inneren Klassen in C++.
In C\# existiert überdies die Möglichkeit, auf als {\it protected} oder {\it privat} deklarierte Attribute zuzugreifen.

\section{In D}

In der Sprache D, welche 1999 ebenfalls als Weiterentwicklung von C++ entwickelt wurde, existiert das gleiche Konzept der inneren Klassen.
Es besteht bei als statisch markierten inneren Klassen nur die Möglichkeit auf als statisch deklarierte Attribute der äußeren Klasse zuzugreifen.
Jedoch wird nicht unterschieden, ob die Klasse innerhalb einer Methode oder Klasse deklariert wurde.

\section{In Pyton und Ruby}

Die Idee des Definierens von Klassen innerhalb anderer Klassen existiert auch in Ruby und Python.
Dabei unterscheiden sich die eingebetteten Klassen in diesen Sprachen jedoch maximal durch den Namensraum,
über den Sie angesprochen werden können, von den "'normalen"' Klassen.
