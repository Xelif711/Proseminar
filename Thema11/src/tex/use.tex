\chapter{Verwandte Konzepte in anderen Sprachen}
\section{In C++}

Das Konzept der inneren Klassen wurde erstmals 1989 in C++ eingeführt.
Diese sind sehr ähnlich zu Javas heutigen statischen inneren Klassen, wobei der größte Unterschied in der Möglichkeit Javas besteht, auf private, für andere Klassen nicht einsehbare Attribute zuzugreifen, was in C++ nicht möglich ist.
Also sind Javas innere Klassen eine Art Erweiterung der inneren Klassen von C++.
In C++11 wurden die inneren Klassen dann um die Möglichkeit einer "'normalen"' inneren Klasse Javas,
 nämlich auch auf nicht statische Attribute der äußeren Klasse zugreifen zu können, erweitert (\cite{Ellis2007}).
Es existiert auch die Möglichkeit, eine namenlose und somit anonyme Klasse oder {\it struct} zu erstellen, jedoch
ist dies nur in Verbindung mit einem {\it typedef} möglich. Anonmye Klassen sind in C++ auch noch nützlich, um den Verweis auf einen Klasseninhalt so aussehen zu lassen, als ob das
referenzierte Objekt in der gleichen Klasse liegen würde.

\section{In C\#}

Derartige Strukturen existieren auch in C\#, welche eine von Microsoft produzierte Weiterentwicklungen von C++ ist.
Damit folgen diese auch fast den gleichen Regeln wie die inneren Klassen in C++.
In C\# existiert überdies die Möglichkeit, auf als {\it protected} oder {\it privat} deklarierte Attribute zuzugreifen.
In C\# existiert zusätzlich die Möglichkeit, eine namenlose
Klasse zu definieren, welche direkt von {\it Object} erbt. Diese anonyme Klasse hat ähnliche Eigenschaften wie die anonymem Klassen Javas. Sie werden hauptsächlich genutzt,
um "'einen Satz schreibgeschützter Eigenschaften in einem Objekt zu kapseln, ohne zuerst explizit einen Typ für diese Daten definieren zu müssen"'. (\cite{Microsoft:Csharp})

\newpage
\section{In D}

In der Sprache D, welche 1999 ebenfalls als Weiterentwicklung von C++ entwickelt wurde, existiert das gleiche Konzept der inneren Klassen.
Es besteht bei als statisch markierten
inneren Klassen nur die Möglichkeit, auf als statisch deklarierte Attribute der äußeren
Klasse zuzugreifen. Jedoch wird nicht unterschieden, ob die Klasse innerhalb einer Methode oder Klasse deklariert wurde.
Es existiert zusätzlich noch die Möglichkeit, anonyme Klassen zu deklarieren, welche sehr große Ähnlichkeiten zu Javas anonymen Klassen besitzen.(\cite{D:langSpec})

\section{In Python und Ruby}

Die Idee des Definierens von Klassen innerhalb anderer Klassen existiert auch in Ruby und Python.
Dabei unterscheiden sich die eingebetteten Klassen in diesen Sprachen jedoch maximal durch den Namensraum,
über den Sie angesprochen werden können, von den "'normalen"' Klassen.
