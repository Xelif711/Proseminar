\section{Weiteres}

Das Konzept der Inneren Klassen wurde erstmals 1989 in C++ eingeführt. Diese waren semantisch Äquivalent zu Javas Statischen Inneren Klassen.
In C++11 wurden die Inneren Klassen in C++ dann um die Möglichkeiten einer "'Normalen"' Inneren Klasse Javas erweitert (\cite{Ellis2007}).
Selbiges Konzept existiert auch in C\# und Visual Basic, welche beides Weiterentwicklugen Microsofts von C++ sind,
 und somit den Gleichen Regeln folgen wie die Inneren Klassen in C++.

In der Sprache D, welche 1999 als Weiterentwicklung von C++ entwickelt wurde, existiert das gleiche Konzept der Inneren Klassen.

Das Konzept des definieren von Klassen innerhalb anderer Klassen existiert auch in Ruby und Phyton,
 jedoch unterscheiden sich die eingebetteten Klassen in de Sprache maximal durch den Namensraum,
 über den Sie Angesprochen werden können, von "'Normalen"' Klassen der Sprachen.
