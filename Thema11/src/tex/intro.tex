\chapter{Einführung}
\section{Idee / Herkunft}

Eingebettete Klassen wurden in Java 1.1 eingeführt. \cite{Oracle:JDK_Doku1.1.4}
Sie wurden hauptsächlich eingeführt, da die Java-Designer erkannt haben, dass die Sprache eine Art "'Funktionszeiger"' benötigt, welcher jedoch mit der Sprachkonstruktion von Java nicht zu realisieren ist.
Diese Notwendigkeit kam unter anderem von der Existenz der Funktionszeiger in anderen Programmiersprachen, wie z.B. C oder Lisp.
Auch andere Konzepte wie Callbacks oder Adapterklassen lassen sich elegant mit dem Werkzeug der eingebetteten Klassen lösen.

Es gibt insgesamt vier verschiedene Arten von eingebetteten Klassen: Die {\it inneren} {\it Klassen}, die {\it statischen} {\it inneren} {\it Klassen}, die {\it lokalen} {\it Klassen}
und die {\it anonymen} {\it Klassen}.
Diese unterscheiden sich jeweils in der Implementierung, in der Datenkapselung und in den möglichen Funktionalitäten.
Aufgrund der semantischen Nähe werden die statischen inneren Klassen sowie die inneren Klassen jedoch zusammen vorgestellt.

\newpage
