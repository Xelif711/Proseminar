\section{Anonyme Klassen}
\subsection {Implementierung}

{\Large \bf Anonyme Klassen}

Eine {\it anonyme Klasse} wird in Java nach Konstruktoraufruf einer beliebigen Klasse definiert (Listing \ref{lst:ImAnn}). Dabei ist es aufgrund der Klassenerweiterung auch möglich, ein Interface zu instanziieren.
Dazu wird nach dem Aufruf ein Codeblock mit geschweiften Klammern geöffnet, in dem die instanziierte Klasse beliebig um Methoden und Datenfelder erweitert werden kann, welche nicht statisch sein dürfen.
Da die anonyme Klasse von der eigentlich instanziierten Klasse erbt und somit eine Unterklasse dieser ist, können dabei auch Methoden überschrieben werden.
Die konkrete Instanz kann jedoch selbst nur auf schon in der Oberklasse deklarierte Methoden und Variablen zugreifen, da das Objekt der anonymen Klasse automatisch zu einem Objekt der Oberklasse gecastet wird, welche jedoch selbst, wenn sie überschrieben werden, auch auf die Methoden der anonymen Klasse zugreifen dürfen.
Da diese eigentlich neue Klasse keinen Namen zugewiesen bekommen hat, wird sie anonym genannt.
Aufgrund des einmaligen, instanzgebundenen Auftretens einer anonymen Klasse sind weitere Modifikatoren weder sinnvoll noch möglich (\cite{goll2013java}) (\cite{Oracle:JLS9}).

\begin{figure}[hbt]
\lstset{language=Java}
\lstinputlisting[
label=lst:ImAnn,
escapechar=|,
caption=Beispielimplementierung und -instanziierung einer Anonymen Klasse in Java] {src/listings/bsp_ImAnn.java}
\end{figure}

{\bf Sichtbarkeit \& Zugriff}

Es liegt in der Natur der anonymen Klassen, dass auf sie lediglich über die mit ihr verbundenen Instanz zugegriffen werden kann.
Weiterhin kann die anonyme Klasse auch nur mit ihren eigenen Attributen bzw. denen der Oberklasse arbeiten.

\subsection{Vorteile / Nutzen}

Anonyme Klassen werden hauptsächlich und im Idealfall nur benutzt, wenn man eine modifizierte Instanz einer Klasse bzw. eines Interfaces nur einmal gebraucht wird und somit das Übertragen der Modifikationen in eine (innere) Klasse oder Unterklasse nicht sinnvoll wäre.
Außerdem werden sie benutzt, um Instanzen von abstrakten Klassen zu erzeugen, da man direkt die abstrakten Methoden überschreiben kann, ohne vorher eine neue Unterklasse der entsprechenden abstrakten Klasse zu deklarieren.
Weiterhin hilft die Benutzung von anonymen Klassen der Übersichtlichkeit, Lesbarkeit und Verständnis des Codes, da Veränderungen an den Klassen direkt an den benötigten Stellen durchgeführt werden.
Wenn die Modifikationen jedoch in verschiedenen Klassen bzw. Methoden gebraucht werden, sollte jedoch auf eine innere oder lokale Klasse ausgewichen werden, um die Lesbarkeit des Codes zu erhöhen.
\newpage
\subsection{Anwendung / Best Practices}

{\bf Beispiel: Eine einfache GUI mit inneren / anonymen Klassen}

In diesem einfachen Anwendungsbeispiel soll die Anwendung in einem Graphical User Interface (GUI) modelliert werden, die ereignisgesteuert ist.
Es gibt zwei Buttons, die nach Drücken etwas tun sollen.
Die Ereignisbehandlung an sich wird hierbei durch Elemente des Abstract Windows Toolkit (aus jawa.awt) geregelt.
Um nun die Ereignisbehandlung zu implementieren, brauchen wir eine Klasse, welche das ActionListener-Interface aus dem Toolkit implementiert und die Methode dem Aufrufer entsprechend ändert.
Insgesamt benötigen wir also 3 Klassen: Die {\it Anwendung} (PressButtons), die {\it GUI} (GUI) und die {\it Ereignisbehandlung} (MyEventHandler). Dabei wird die Ereignisbehandlung nur von der GUI benutzt, die GUI wiederum nur von der Anwendung. Dabei benutzt die Ereignisbehandlung Methoden, die in der Anwendung aufgerufen werden müssen.
Die Klassen sind also semantisch stark miteinander verbunden und brauchen teilweise vollen Zugriff aufeinander.
Aus den Umständen folgt, dass hier eine entsprechende Schachtelung der Klassen von Vorteil wäre, was sich sehr gut mit den inneren Klassen realisieren lässt.
Dabei wird die Ereignisbehandlung eine innere Klasse des GUI, die ihrerseits eine innere Klasse der Anwendung wird (Listing \ref{lst:AnwAnn}).
So lässt sich der Code prägnant an der Stelle formulieren, wo er gebraucht wird.
Währenddessen sind die betroffenen Klassen sonst komplett unsichtbar.
Doch hier können auch die anonymen Klassen genutzt werden, denn momentan agiert nur eine Instanz des EventHandlers, die je nach Button einen Unterschied macht. Es kann jedoch deutlich eleganter sein, mehrere "'Einweginstanzen"' zu haben, die dann nur auf einen Button reagieren und dementsprechend nur eine Methode überschreiben müssen.
Dadurch fällt der Overhead durch if-Cases weg und man kann noch klarer sehen, was durch das Drücken eines bestimmten Buttons ausgelöst werden soll (Listing \ref{lst:AnwAnnB}). Weiterhin kann man sich das Schreiben einer weiteren Klasse sparen.
\\
\\

\begin{figure}[H]
\lstset{language=Java}
\lstinputlisting[
label=lst:AnwAnnB,
escapechar=|,
caption=Mögliche GUI-Implementierung mit anonymen statt Inneren Klassen] {src/listings/bsp_AnwAnnB.java}
\end{figure}
\newpage

\begin{figure}[H]
\lstset{language=Java}
\lstinputlisting[
label=lst:AnwAnn,
escapechar=|,
caption=Mögliche GUI-Implementierung mit inneren Klassen] {src/listings/bsp_AnwAnn.java}
\end{figure}
\newpage

\section{Vergleich der Konzepte}

Eingebettete Klassen lassen sich als eine von Javas vielen Ordnungsstrukturen insgesamt wohl am besten auf den gemeinsamen Vorteil der verbesserten Datenkapselung reduzieren, während andere Vorteile eher spezieller Natur sind.
Dies hat den einfachen Grund, dass alle eingebetteten Klassen die implementierten Funktionalitäten möglichst lokal benutzen. Die innere Klassen sollen nur der äußeren Klasse zur Verfügung stehen,
eine lokale Klasse kann nur von ihrer umschließenden Methode benutzt werden und die anonyme Klasse gilt nur für eine Instanz. Dass dies jedoch ein zweischneidiges Schwert ist, zeigt der gemeinsame Nachteil:
Es wird sich darauf festgelegt, eine Klasse nur lokal zu verwenden - besteht jedoch im Nachgang der Bedarf, diese Klasse auch an anderer Stelle zu benutzen. In diesem Fall muss unter Umständen viel im Code geändert werden.
Dabei sind die "'sperrigeren"' Konzepte wie die inneren und lokalen Klassen offenbar stärker betroffen als die flexiblen anonymen Klassen.
Dies führt nach unserem Erachten auch dazu, dass die Einsatzbereiche innerer und lokaler Klassen stärker eingegrenzt und eher Spezialfälle sind, während der Gebrauch einer anonymen Klasse
schnell sowie unkompliziert ist und daher stets in Betracht gezogen werden kann. Wiederum scheint es eher wenig Situationen zu geben, in denen besonders der Einsatz einer lokalen Klasse sinnvoll erscheint.
