\chapter{Eingebettete Klassen}
\section{(Statische) Innere Klassen}
\subsection {Implementierung / Instanziierung}

\begin{figure}[H]
\lstset{language=Java}
\lstinputlisting[
label=lst:ImpInn,
escapechar=|,
caption=Beispielimplementierung von inneren Klassen in Java] {src/listings/bsp_ImpInn.java}
\end{figure}
\newpage
Eine innere Klasse wird in Java durch das Erstellen einer Klasse innerhalb eines bestehenden Klassenblocks implementiert.
Der Name, der inneren Klasse, muss unterschiedlich sein. Dabei kann diese mit allen Modifikatoren versehen werden, die auch einer äußeren Klasse zur Verfügung stehen - sie darf aber auch {\it private}, {\it protected} oder {\it static} sein, da sie letzten Endes ein Attribut der äußeren Klasse ist.
Obwohl klassische Mehrfachvererbung (wie in C++) in Java nicht möglich ist, ist es inneren Klassen erlaubt, von einer weiteren Klasse (ausgenommen ihrer äußeren Klasse) zu erben.
Falls es gleichnamige Attribute gibt, wird das Attribut der äußeren Klasse vom Attribut der Oberklasse, von welcher die innere Klasse erbt, verdeckt.
Außerdem können in einer inneren Klasse Attribute implementiert werden. Diese verdecken wiederum gleichnamige Attribute der Ober- oder äußeren Klasse.
Bei einer nicht statischen inneren Klasse erlaubt Java keine statischen Attribute, da die innere Klasse an sich ein Attribut der äußeren Klasse ist und sonst ein nicht statisches Attribut statische Elemente enthalten würde.
Falls die Klasse statisch ist, verhält sich diese ähnlich einer normalen Klasse: Es dürfen statische sowie instanzgebundene Attribute verwendet werden, da die Klasse, obwohl sie eigentlich ein statisches Attribut ist, instanziiert werden kann (Listing \ref{lst:ImpInn}).


{\bf Instanziierung}

Eine normale innere Klasse ist an eine Instanz ihrer äußere Klasse gebunden. Bevor der Konstruktor der inneren Klasse aufgerufen werden kann, muss also zuerst eine Instanz der äußeren Klasse generiert werden.
Dann wird der Konstruktor der inneren Klasse von der Instanz der äußeren Klasse aus aufgerufen.
Im Gegensatz dazu benötigt eine statische innere Klasse keine Instanz der äußeren Klasse.
In diesem Fall darf einfach der Konstruktor der statischen inneren Klasse über den Namensraum der äußeren Klasse aufgerufen werden.
In beiden Fällen werden die Objektdeklarationen über den Namensraum der äußeren Klasse geregelt (Listing \ref{lst:InstInn}) \cite{goll2013java}.

Dies erklärt sich durch Speicherung des Bytecodes der inneren Klassen, welche nach
der Kompilierung als .class Datei im gleichen Ordner wie die äußere Klasse abgelegt
wird. Jedoch wird sie mit {\it ÄußereKlasse\$InnereKlasse.class} bezeichnet, und nicht in der gleichen Datei
gespeichert wie die äußere Klasse.

\begin{figure}[hbt]
\lstset{language=Java}
\lstinputlisting[
label=lst:InstInn,
escapechar=|,
caption=Beispielinstanziierung von inneren Klassen in Java] {src/listings/bsp_InstInn.java}
\end{figure}

{\bf Sichtbarkeit \& Zugriff}

Generell gilt, dass die innere (statische) Klasse je nach Modifikator genau wie ein Attribut der äußeren Klasse gesehen wird, was dann implizit auch für alle Attribute der inneren Klasse gilt, falls diese einer weniger starken Modifikator benutzen.
Auch muss stets der Namensraum der äußeren Klasse benutzt werden, sollte auf Attribute der inneren Klasse zugegriffen werden.
Weiterhin hat die innere Klasse im Normalfall Zugang zu allen Methoden und Variablen der äußeren Klasse - sogar zu denen, die als private deklariert wurden.
Dies gilt andersherum ebenso: Auch die äußere Klasse kann auf die komplette innere Klasse zugreifen, gleich ob diese privat ist oder nicht.
Eine große Unterscheidung muss jedoch gemacht werden, wenn eine innere Klasse statisch ist. Da in diesem Fall keine Instanz der äußeren Klasse verbunden ist, können auch lediglich statische Attribute der äußeren Klasse benutzt werden (Listing \ref{lst:SZInn}).
\\
\\
\begin{figure}[hbt]
\lstset{language=Java}
\lstinputlisting[
label=lst:SZInn,
caption=Sichtbarkeitsbeispiel für innere Klassen in Java] {src/listings/bsp_SZ_Inn.java}
\end{figure}

\subsection{Vorteile und Nutzen}

Ein Grund für die Benutzung der inneren Klassen ist die Möglichkeit, innerhalb der inneren Klasse auf die privaten Attribute der äußeren Klasse zugreifen und modifizieren zu können.
Dies ist insbesonders ein Vorteil für die Datenkapselung, da innerhalb der inneren Klasse mit Hilfe von Funktionen der Oberklasse alle Attribute der äußeren Klasse verändert werden können, ohne dass die Oberklasse (und somit weitere Klassen) Zugriff auf diese Daten haben.

Ein Grund dafür ist die Möglichkeit, dass eine innere Klasse, welche implizit die Methoden und Variablen der äußeren Klasse erbt, eine weitere Oberklasse erweitern kann und somit auch deren Variablen und Methoden erbt.
Zudem können Klassen, welche von der inneren Klasse erben (was möglich ist, wenn diese nicht als {\it private}, {\it protected} oder {\it final} markiert sind) über die innere Klasse auf die Attribute und Methoden der äußeren Klasse zugreifen, ohne dass diese direkt von ihr erbt. Dies kann jedoch auch zu Problemen führen.

Es entsteht dadurch die Möglichkeit, dass es durch die Sichtbarkeit der inneren Klasse nach außen zu zyklischen Vererbungen kommt. Dies bedeutet, dass eine Klasse (indirekt) von sich selbst erbt und somit eventuell eigene Methoden überschreibt (\cite{DBLP:journals/corr/abs-1301-6260}) (Listing \ref{lst:ZclInn}).
Im Beispiel \ref{lst:ZclInn}  wird die zyklische Vererbung nicht erkannt, da die Klasse {\it innerA} die Klasse {\it A} nicht erweitert und somit nicht direkt von ihr erbt. Damit kann der Java-Compiler die Tatsache, dass {\it innerA} durch {\it A} von {\it B} erbt, nicht erkennen und der Code wird fehlerlos kompiliert.
Dies kann, wenn es nicht bemerkt wird, zu falschem und aus der Sicht der Entwickler, sonderbarem Verhalten führen, obwohl die Implementierung semantisch korrekt ist.
Um die Möglichkeit zu verringern, dass eine zyklische Vererbung auftreten kann, wurde das sogenannte PICIP (Eine Kurzform für "'Potential Inner Class Inheritance Problem"') enwickelt, welches das Potential beschreibt, mit dem Vererbungsprobleme bei inneren Klassen auftreten könnten.
\\
\\
\begin{figure}[hbt]
\lstset{language=Java}
\lstinputlisting[
label=lst:ZclInn,
escapechar=|,
caption=Von Java nicht erkannte Zyklische Vererbung] {src/listings/bsp_Zcl_A.java}
\end{figure}

Weiterhin wird durch Einsatz innerer Klassen die Implementierung einer Adapterklasse erleichtert, welche genau an der Stelle mit den entsprechenden Methoden einer externen Klasse deklariert werden kann.

Man kann durch die Deklaration einer benötigten Klasse als innere Klasse an der entsprechenden Stelle die Lesbarkeit/Übersichtlichkeit des Codes erhöhen.
Dadurch ist es jedoch auch möglich, dass der Code durch zu viele bzw. zu tief ineinander verschachtelte innere Klassen unübersichlich und somit unangenehm zu lesen wird.
Allgemein gilt daher, dass so wenig innere Klassen wie möglich ineinander geschachtelt werden sollten, um besonders die Fehlererkennung zu verbessern.

Namenskonflikte mit anderen Klassen im Paket können außerdem vermieden werden, da mehrere Klassen mit gleichem Namen existieren können, aber
die (statischen) inneren Klassen jedoch über ihre äußere Klasse instanziiert werden müssen und es somit nicht zu Verwechslungen und Konflikten mit gleichnamigen Klassen im Paket kommen kann.

\subsection{Anwendungen / Best Practices}

Innere Klassen sollten hauptsächlich dann verwendet werden, wenn eine Klasse ausschließlich von der Oberklasse benutzt wird.
Ein weiterer Anwendungsfall wäre die Ausnutzung der Mehrfachvererbung oder die Möglichkeit, neuen Methoden Zugriff auf die privaten Attribute der äußeren Klasse zu erlauben.

{\bf Beispiel \ref{lst:AnwInn}: Der Ladeadapter}

In diesem simplen Beispiel geht es darum, einen Adapter für den Ladevorgang eines Handys der Marke A zu bauen. Dabei benutzen wir eine Adapterklasse, welche ein Strukturmuster aus der Softwaretechnik ist. Wenn eine Klasse Methoden einer anderen Klasse benötigt, jedoch nur ein Objekt hat, welches ein Interface implementiert, so kann man eine Adapterklasse benutzen.
Wir benötigen für das HandyA ein Objekt, welches das Interface {\it Ladegeraet} implementiert.
Wir haben jedoch kein passendes Ladegeraet von A zur Verfuegung, sondern nur ein Ladegeraet einer Marke B, welches das Interface nicht implementiert.
Da jedoch auch dieses im Prinzip laden kann, versuchen wir, diese Funktionalität für unser Handy zu nutzen.
Dies können wir durch die Einbindung der innere Klasse Adapter erreichen, welche das Interface implementiert.
Jedoch steht uns hier auch das Ladegeraet der Marke B zur Verfügung.
Dieses instanziieren wir und überschreiben die Methode des Interfaces dadurch, dass wir hier die Lademethode des Handy B nutzen.
Damit können wir eine Instanz des Adapters als Ladegeraet für unser Handy nutzen, obwohl uns kein "'korrektes"' Ladegeraet zur Verfügung stand.
\\
\begin{figure}[hb]
\lstset{language=Java}
\lstinputlisting[
label=lst:AnwInn,
escapechar=|,
caption=Beispielanwendung von inneren Klassen in einem Adapter] {src/listings/bsp_AnwInn.java}
\end{figure}


\newpage
